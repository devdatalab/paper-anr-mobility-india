%%%%%%%%%%%%%%%%%%%%%%%%%%
%% Transition matrices  %%
%%%%%%%%%%%%%%%%%%%%%%%%%%
\begin{table}[H]
  \thisfloatpagestyle{empty}
  \caption{Transition Matrices for Father and Son Education in India}
\vspace{-.5cm}
  \label{tab:trans_matrices}
        {\scriptsize

          \begin{center}
            \textbf{A. Sons Born 1950-59} 
          \end{center}

          \begin{center}
            \begin{tabular}{c |c c c c c c c}
              \hline
              \hline
              & \multicolumn{7}{c}{Son highest education attained} \\
              \tiny  & $<$ 2 yrs. & 2-4 yrs. & Primary & Middle & Sec. & Sr. sec. & Any higher \\Father ed attained & (31\%) &  (11\%) & (17\%) & (13\%) &  (13\%) &  (6\%) &  (8\%) \\ 
              \hline
              \tiny $<$2 yrs. (58\%) & 0.46 & 0.12 & 0.16 & 0.11 & 0.09 & 0.03 & 0.03 \\ 
2-4 yrs. (12\%) & 0.10 & 0.18 & 0.21 & 0.19 & 0.16 & 0.09 & 0.07 \\ 
Primary (13\%) & 0.07 & 0.08 & 0.31 & 0.16 & 0.19 & 0.08 & 0.10 \\ 
Middle (5\%) & 0.06 & 0.05 & 0.09 & 0.30 & 0.18 & 0.14 & 0.18 \\ 
Secondary (5\%) & 0.03 & 0.01 & 0.04 & 0.13 & 0.38 & 0.11 & 0.30 \\ 
Sr. secondary (2\%) & 0.02 & 0.00 & 0.02 & 0.10 & 0.11 & 0.36 & 0.38 \\ 
Any higher ed (2\%) & 0.01 & 0.01 & 0.01 & 0.03 & 0.09 & 0.13 & 0.72 \\ 
 
              \hline 


            \end{tabular}
          \end{center}


          \begin{center}
            \textbf{B. Sons Born 1960-69} 

          \end{center}

          \begin{center}
            \begin{tabular}{c |c c c c c c c}

              \hline
              \hline
              & \multicolumn{7}{c}{Son highest education attained} \\
              \tiny  & $<$ 2 yrs. & 2-4 yrs. & Primary & Middle & Sec. & Sr. sec. & Any higher \\Father ed attained & (27\%) &  (10\%) & (16\%) & (16\%) &  (14\%) &  (7\%) &  (10\%) \\ 
              \hline
              \tiny $<$2 yrs. (56\%) & 0.40 & 0.12 & 0.16 & 0.14 & 0.10 & 0.04 & 0.03 \\ 
2-4 yrs. (13\%) & 0.11 & 0.16 & 0.18 & 0.23 & 0.15 & 0.07 & 0.08 \\ 
Primary (13\%) & 0.10 & 0.05 & 0.25 & 0.18 & 0.19 & 0.09 & 0.14 \\ 
Middle (5\%) & 0.06 & 0.04 & 0.08 & 0.28 & 0.21 & 0.13 & 0.19 \\ 
Secondary (6\%) & 0.03 & 0.02 & 0.08 & 0.11 & 0.35 & 0.17 & 0.25 \\ 
Sr. secondary (2\%) & 0.02 & 0.01 & 0.03 & 0.07 & 0.20 & 0.25 & 0.42 \\ 
Any higher ed (2\%) & 0.01 & 0.00 & 0.02 & 0.02 & 0.08 & 0.11 & 0.75 \\ 
 
              \hline 
            \end{tabular}
          \end{center}

          \begin{center}
            \textbf{C. Sons Born 1970-79} 
          \end{center}

          \begin{center}
            \begin{tabular}{c |c c c c c c c}

              \hline
              \hline
              & \multicolumn{7}{c}{Son highest education attained} \\
              \tiny  & $<$ 2 yrs. & 2-4 yrs. & Primary & Middle & Sec. & Sr. sec. & Any higher \\Father ed attained & (20\%) &  (7\%) & (16\%) & (17\%) &  (16\%) &  (10\%) &  (14\%) \\ 
              \hline
              \tiny $<$2 yrs. (49\%) & 0.34 & 0.09 & 0.19 & 0.16 & 0.13 & 0.05 & 0.04 \\ 
2-4 yrs. (11\%) & 0.10 & 0.13 & 0.18 & 0.24 & 0.18 & 0.08 & 0.10 \\ 
Primary (12\%) & 0.08 & 0.06 & 0.22 & 0.19 & 0.18 & 0.12 & 0.14 \\ 
Middle (6\%) & 0.05 & 0.01 & 0.07 & 0.27 & 0.21 & 0.17 & 0.21 \\ 
Secondary (8\%) & 0.04 & 0.01 & 0.05 & 0.11 & 0.28 & 0.19 & 0.32 \\ 
Sr. secondary (2\%) & 0.01 & 0.01 & 0.02 & 0.10 & 0.17 & 0.21 & 0.48 \\ 
Any higher ed (4\%) & 0.00 & 0.00 & 0.02 & 0.03 & 0.10 & 0.14 & 0.70 \\ 
 
              \hline 
            \end{tabular}
          \end{center}

          \begin{center}
            \textbf{D. Sons Born 1980-89} 
          \end{center}

          \begin{center}
            \begin{tabular}{c | c c c c c c c}
              \hline
              \hline
              & \multicolumn{7}{c}{Son highest education attained} \\
              \tiny  & $<$ 2 yrs. & 2-4 yrs. & Primary & Middle & Sec. & Sr. sec. & Any higher \\Father ed attained & (13\%) &  (6\%) & (16\%) & (23\%) &  (15\%) &  (11\%) &  (16\%) \\ 
              \hline
              \tiny $<$2 yrs. (35\%) & 0.25 & 0.10 & 0.21 & 0.24 & 0.10 & 0.05 & 0.04 \\ 
2-4 yrs. (10\%) & 0.09 & 0.11 & 0.18 & 0.28 & 0.15 & 0.10 & 0.09 \\ 
Primary (14\%) & 0.05 & 0.05 & 0.24 & 0.26 & 0.17 & 0.10 & 0.13 \\ 
Middle (9\%) & 0.03 & 0.03 & 0.07 & 0.32 & 0.19 & 0.17 & 0.19 \\ 
Secondary (9\%) & 0.01 & 0.00 & 0.05 & 0.16 & 0.25 & 0.22 & 0.30 \\ 
Sr. secondary (4\%) & 0.01 & 0.01 & 0.04 & 0.09 & 0.16 & 0.22 & 0.47 \\ 
Any higher ed (5\%) & 0.01 & 0.00 & 0.02 & 0.08 & 0.14 & 0.14 & 0.62 \\ 
 
              \hline 
            \end{tabular}
          \end{center}
        }
\vspace{-.4cm} 
        \footnotesize{Table \ref{tab:trans_matrices} shows transition matrices by decadal birth
        cohort for Indian fathers and sons. These data are visualized in Figure~\ref{fig:rank_scatters} for all father/mother-son/daughter dyads.  Source: IHDS (2012).}

\end{table}

%%%%%%%%%%%%%%%%%%%%%%%%%%%%%%%%%%%
%% CONSISTENCY OF ED MEASUREMENT %%
%%%%%%%%%%%%%%%%%%%%%%%%%%%%%%%%%%%
\begin{table}[H]
  \caption{Internal Consistency of Reports of Parents' Education}
  \label{tab:validate_eds}
  \setlength{\linewidth}{.1cm} \begin{center}
\newcommand{\contents}{\begin{tabular}{l*{6}{c}}
\hline\hline
 & \multicolumn{2}{c}{Father-Son} & \multicolumn{2}{c}{Father-Daughter} & \multicolumn{2}{c}{Mother-Daughter} \\ 
%PYTHON_HEADER
                    &         (1)   &         (2)   &         (3)   &         (4)   &         (5)   &         (6)   \\
\hline
Age                 &               &      -0.000   &               &      -0.018   &               &      -0.008   \\
                    &               &     (0.008)   &               &     (0.016)   &               &     (0.007)   \\
Child years of education&               &       0.008   &               &       0.037*  &               &       0.003   \\
                    &               &     (0.013)   &               &     (0.021)   &               &     (0.011)   \\
Log household income&               &      -0.005   &               &      -0.051   &               &      -0.026   \\
                    &               &     (0.029)   &               &     (0.058)   &               &     (0.036)   \\
Constant            &       0.053   &       0.054   &      -0.002   &       0.912   &       0.006   &       0.545   \\
                    &     (0.056)   &     (0.431)   &     (0.103)   &     (0.841)   &     (0.052)   &     (0.466)   \\
N                   &        1258   &        1255   &         440   &         440   &         726   &         725   \\
r2                  &        0.00   &        0.00   &        0.00   &        0.01   &        0.00   &        0.00   \\
\hline
\multicolumn{7}{p{\linewidth}}{$^{*}p<0.10, ^{**}p<0.05, ^{***}p<0.01$} \\
\multicolumn{7}{p{\linewidth}}{\footnotesize \tablenote}
\end{tabular} }
\setbox0=\hbox{\contents}
\setlength{\linewidth}{\wd0-2\tabcolsep-.25em} \contents \end{center}


    \footnotesize{Table~\ref{tab:validate_eds} shows measures of
      internal consistency when there are multiple reports of an
      individual's father in the IHDS. We calculate the difference between a person's report of their parent's education and the parent's own reporting of it when in the same household. We then regress this difference on a constant (which provides the average difference, in Columns 1, 3, and 5), and on a series of household characteristics (Columns 2, 4, and 6). Source: IHDS (2012).}
\end{table}

%%%%%%%%%%%%%%%%%%%%%%%%%%%%%%%%%%%%%%%%%%%
%% Stats About People in the Bottom Half %%
%%%%%%%%%%%%%%%%%%%%%%%%%%%%%%%%%%%%%%%%%%%
\begin{landscape}
\begin{table}[H]
  \caption{Characteristics of Top and Bottom Half Individuals and Households}
  \label{tab:bottom_half_stats}
  \begin{center}
  \small \begin{tabular}{llllll}
\hline
           &                             & \multicolumn{2}{c}{Age 20--29}       & \multicolumn{2}{c}{Age 50--59}                                                                                     \\
           &                             & Bottom Half                          & Top Half                             & Bottom Half                          & Top Half                             \\ \hline
Individual & Any wage                    & 0.712          & 0.417          & 0.636          & 0.561          \\
           &                             & (0.453)     & (0.493)     & (0.481)     & (0.496)     \\
           & Log(wage)                   & 2.875           & 3.237           & 2.969           & 3.785           \\
           &                             & (0.510)      & (0.699)      & (0.609)      & (0.947)      \\
           & Rural                       & 0.738             & 0.548             & 0.762             & 0.483             \\
           &                             & (0.440)        & (0.498)        & (0.426)        & (0.500)        \\
           & Years of Education          & 4.988          & 12.167          & 1.850          & 10.487          \\
           &                             & (3.017)     & (1.647)     & (2.170)     & (2.254)     \\
           & Muslim                      & 0.176            & 0.097            & 0.127            & 0.060            \\
           &                             & (0.381)       & (0.296)       & (0.333)       & (0.238)       \\
           & SC                          & 0.259                & 0.196                & 0.240                & 0.130                \\
           &                             & (0.438)           & (0.397)           & (0.427)           & (0.336)           \\
           & ST                          & 0.120                & 0.053                & 0.121                & 0.047                \\
           &                             & (0.325)           & (0.225)           & (0.326)           & (0.211)           \\
           &                             &                                      &                                      &                                      &                                      \\
Household  & Log(income)                 & 10.809      & 11.372      & 11.102      & 11.810      \\
           &                             & (0.833) & (0.978) & (0.942) & (1.062) \\
           & Log(per capita consumption) & 9.654           & 10.133           & 9.764           & 10.301           \\ 
           &                             & (0.554)      & (0.641)      & (0.606)      & (0.677)      \\ \hline
\end{tabular}%


  \end{center}

  \footnotesize{Table~\ref{tab:bottom_half_stats} shows summary statistics describing individuals from the bottom and top half of the education distribution, respectively. The individual statistics describe men born in 1983--92, and in 1953--62. The household statistics describe the households where those men reside. Standard deviations are in parentheses.}

\end{table}
\end{landscape}

%%%%%%%%%%%%%%%%%%%%
%% Granular bins  %%
%%%%%%%%%%%%%%%%%%%%
\begin{table}[H]
  \caption{Bottom Half Mobility Calculated Using Binned vs. Granular Education}
  \label{tab:gran}
\begin{center}
Panel A: Binned Education
\end{center}
\begin{center}
  \begin{tabular}{ccc} 
\hline
\hline
Group & 1960--69 & 1980--89 \\
\hline 
All & [36.6, 39.0] &
[37.1, 37.2] \\
Forward/Other & [41.8, 44.0] &
[41.3, 41.3] \\
Muslim & [31.3, 33.6] & [28.9, 29.0] \\
Scheduled Castes & [32.9, 35.2] & [36.9, 37.0] \\
Scheduled Tribes & [29.4, 31.3] & [33.1, 33.1] \\
\hline
\hline 
\end{tabular}

\end{center}
\begin{center}
Panel B: Granular Education
\end{center}
\begin{center}
  \begin{tabular}{ccc} 
\hline
\hline
Group & 1960--69 & 1980--89 \\
\hline 
All & [36.5, 38.9] &
[36.3, 37.2] \\
Forward/Other & [41.6, 43.7] &
[41.1, 41.1] \\
Muslim & [31.2, 33.6] & [28.1, 29.3] \\
Scheduled Castes & [33.0, 35.2] & [36.5, 37.0] \\
Scheduled Tribes & [29.3, 31.3] & [33.4, 33.5] \\
\hline
\hline 
\end{tabular}

\end{center}
\footnotesize{Table~\ref{tab:gran} compares national
  and subgroup bottom half mobility when calculated using IHDS
  data downcoded to match standard education categories (Panel A, identical to Table~\ref{tab:mob_changes}) and using IHDS data with
  unadjusted granular years of education (Panel B). The results are similar because there are few individuals with education levels which were both in the bottom 50\% and needed to be downcoded.}
\end{table}


%%%%%%%%%%%%%%%%%%%%%
%% FERTILITY TABLE %%
%%%%%%%%%%%%%%%%%%%%%

\begin{table}[H]
  \caption{Relationship Between Fertility and Subgroup Upward Mobility}
  \label{tab:mech_fert}
  \setlength{\linewidth}{.1cm} \begin{center}
\newcommand{\contents}{\begin{tabular}{l*{3}{c}}
\hline\hline
                    &         (1)   &         (2)   &         (3)   \\
\hline
Muslim              &     -13.476***&     -12.338***&      -9.287***\\
                    &     (0.976)   &     (1.697)   &     (1.721)   \\
Scheduled Caste     &      -4.163***&      -2.608** &      -1.901   \\
                    &     (0.749)   &     (1.281)   &     (1.268)   \\
Scheduled Tribe     &      -9.075***&      -8.040***&      -8.291***\\
                    &     (1.076)   &     (1.851)   &     (1.829)   \\
Urban               &       3.881***&       3.812***&       3.514***\\
                    &     (0.782)   &     (1.276)   &     (1.261)   \\
Number of Siblings  &               &               &      -2.359***\\
                    &               &               &     (0.304)   \\
N                   &        6345   &        2347   &        2347   \\
r2                  &        0.11   &        0.15   &        0.18   \\
\hline
\multicolumn{4}{p{\linewidth}}{$^{*}p<0.10, ^{**}p<0.05, ^{***}p<0.01$} \\
\multicolumn{4}{p{\linewidth}}{\footnotesize \tablenote}
\end{tabular} }
\setbox0=\hbox{\contents}
\setlength{\linewidth}{\wd0-2\tabcolsep-.25em} \contents \end{center}

  
  \footnotesize{Table~\ref{tab:mech_fert} shows estimates from
    regressions of child education rank on social group indicators and
    an individual's number of siblings, a proxy for mother's
    fertility. The sample is limited to individuals born in 1985--89
    to fathers with two or fewer years of education. The outcome
    variable thus corresponds to $\mu_0^{51}$, a close analog of
    bottom half mobility ($\mu_0^{50}$). Column 1 shows the estimation
    without the fertility measure for the full sample. Column 2 limits
    the data to the set of individuals for whom mother's fertility can
    be measured, and Column 3 adds the fertility variable. The effect
    of fertility on subgroup mobility gaps is understood as the change
    in the subgroup coefficient from Column 2 to Column 3. All
    regressions control for state fixed effects. Source: IHDS (2012).}
\end{table}

