%%%%%%%%%%%%%%%%%%%%%%%
%% CHANGES OVER TIME %%
%%%%%%%%%%%%%%%%%%%%%%%
\begin{landscape}
\clearpage 
\pagestyle{fancy}
\markright{} 

\begin{table}
  \caption{Changes in Upward Mobility Over Time}
  \label{tab:mob_changes}
  \begin{center}
    \panel{A. Father-son pairs}
  \end{center}
\begin{center}
  \begin{tabular}{cccccc} 
\hline
\hline
& All groups & Forward/Others & Muslims & SCs & STs \\
\hline 
1960--1969 & [36.6, 39.0] &
[41.8, 44.0] &
[31.3, 33.6] &
[32.9, 35.2] & 
[29.4, 31.3] \\ 
& \{35.7, 39.8\} &
\{40.6, 45.3\} &
\{29.4, 35.6\} &
\{31.3, 36.8\} & 
\{27.0, 33.7\} \\ 
& & & & & \\ 
1980--1989 & [37.1, 37.2] &
[41.3, 41.3] &
[28.9, 29.0] &
[36.9, 37.0] & 
[33.1, 33.1] \\ 
& \{36.4, 37.8\} &
\{40.3, 42.3\} &
\{27.7, 30.2\} &
\{35.5, 38.5\} & 
\{31.2, 35.1\} \\ 
& & & & & \\ 
Change over time & [-1.9, 0.6] &
[-2.7, -0.5] &
[-4.7, -2.3] &
[1.8, 4.1] & 
[1.8, 3.7] \\ 
& \{-3.0, 1.7\} &
\{-4.3, 1.1\} &
\{-7.0, -0.0\} &
\{-0.4, 6.3\} & 
\{-1.3, 6.8\} \\ 
Fraction overlapping bounds & 0.818 & 
0.310 & 0.054 & 0.090 & 0.119 \\ 
\hline
\hline 
\end{tabular}
 
\end{center}
  \begin{center}
    \panel{B. Father-daughter pairs}
  \end{center}
\begin{center}
  \begin{tabular}{cccccc} 
\hline
\hline
& All groups & Forward/Others & Muslims & SCs & STs \\
\hline 
1960--1969 & [34.9, 41.0] &
[38.7, 44.8] &
[33.5, 38.9] &
[31.3, 36.8] & 
[31.4, 33.8] \\ 
& \{34.1, 41.8\} &
\{37.6, 45.9\} &
\{31.8, 40.6\} &
\{29.8, 38.3\} & 
\{29.0, 36.2\} \\ 
& & & & & \\ 
1980--1989 & [35.4, 35.5] &
[38.0, 38.2] &
[32.0, 33.5] &
[32.9, 34.2] & 
[30.4, 30.5] \\ 
& \{34.6, 36.3\} &
\{36.8, 39.3\} &
\{30.9, 34.6\} &
\{31.7, 35.4\} & 
\{28.3, 32.6\} \\ 
& & & & & \\ 
Change over time & [-5.6, 0.6] &
[-6.9, -0.5] &
[-6.9, -0.0] &
[-3.9, 2.9] & 
[-3.4, -0.9] \\ 
& \{-6.7, 1.7\} &
\{-8.4, 1.0\} &
\{-8.8, 1.9\} &
\{-5.8, 4.8\} & 
\{-6.6, 2.2\} \\ 
Fraction overlapping bounds & 0.781 & 
0.244 & 0.434 & 0.985 & 0.346 \\ 
\hline
\hline 
\end{tabular}
 
\end{center}
\\ 
\footnotesize{Table~\ref{tab:mob_changes} shows estimates of full
    sample and subgroup bottom half mobility ($\mu_0^{50}$) for the
    1960--69 and 1980--89 birth cohorts for father-son (Panel A) and
    father-daughter (Panel B) pairs. We show both bounds (in square
    brackets) and 90\% confidence sets (in curly braces) on those
    bounds. The table also reports the bounds and 90\% confidence sets
    on the change in bottom half mobility between these two time
    periods. We obtain confidence sets by generating 1,000 bootstrap
    draws, estimating bounds on each bootstrap draw, and following the
    framework in \citeasnoun{cherno2007} to form 90\% confidence
    sets from bootstrapped bounds. Because these are
    confidence sets rather than confidence
    intervals, instead of $p$-values we show the fraction of bootstraps
    in which the 1960--69 and 1980--89 bounds are overlapping. Source:
    IHDS (2012).}
\end{table}
\end{landscape} 
%%%%%%%%%%%%%%%%%%%%%%%
%% GROUP DIFFERENCES %%
%%%%%%%%%%%%%%%%%%%%%%%
\clearpage
\pagestyle{fancy}

\begin{table}[H]
  \caption{Group Differences in Upward Mobility}
  \label{tab:group_diffs}
  \hline
\hline
\begin{tabular}{cccc} 
& F/O minus SC & F/O minus Muslim & SC minus Muslim \\ 
\hline 
Father/son ($\mu_0^{50}$) & [4.6, 5.0] &
                            [11.6, 12.1] &
                            [6.9, 7.3] \\
& \{2.8, 6.8\} &
  \{10.0, 13.8\} &
  \{4.5, 9.6\} \\
Fraction overlapping bounds & 0.000  & 0.000 & 0.000 \\ 
\hline

Father/daughter ($\mu_0^{50}$) & [4.2, 4.5] &
                                 [5.1, 5.5] &
                                 [0.8, 1.1] \\
& \{1.9, 6.8\} &
  \{2.9, 7.7\} &
  \{-2.0, 3.9\} \\
Fraction overlapping bounds & 0.001  & 0.000 & 0.511 \\ 
\hline

Father/son ($\mu_{50}^{100}$) & [4.8, 5.3] &
                                [9.0, 9.4] &
                                [3.9, 4.3] \\
& \{3.3, 6.8\} &
  \{5.8, 12.6\} &
  \{0.8, 7.5\} \\
Fraction overlapping bounds & 0.000  & 0.000 & 0.005 \\ 
\hline

Father/daughter ($\mu_{50}^{100}$) & [7.7, 8.0] &
                                [7.8, 8.2] &
                                [0.0, 0.4]  \\
 & \{4.0, 11.7\} &
   \{5.3, 10.8\} &
   \{-3.9, 4.3\} \\
Fraction overlapping bounds & 0.000  & 0.000 & 0.318 \\ 
\end{tabular}
\hline
\hline 


  \newline
  \newline
  \footnotesize{Table~\ref{tab:group_diffs} shows estimates of
    cross-group differences in bottom half mobility ($\mu_0^{50}$) and
    top half mobility ($\mu_{50}^{100}$) in the 1980--89 birth
    cohorts. F/O stands for Forwards/Others and SC for Scheduled Castes. We show both bounds (in square brackets) and 90\% confidence sets (in curly braces) on those bounds. We obtain confidence sets by generating 1,000 bootstrap draws, estimating
    bounds on each bootstrap draw, and following the framework in
    \citeasnoun{cherno2007} to form 90\% confidence sets from
    bootstrapped bounds. Because these are confidence sets rather than
    confidence intervals, instead of $p$-values we show the fraction
    of the bounds for the two social groups that are overlapping. For
    example, the value of 0.509 in the final column indicates that
    50.9\% of the bootstraps generate overlapping bounds for the two
    groups. Source: IHDS (2012).}
\end{table}

%%%%%%%%%%%%%%%%%%
%% CASSAN TABLE %%
%%%%%%%%%%%%%%%%%%
\begin{table}[H]
  \caption{Effect of Scheduled Caste Designation on Upward Mobility}
  \label{tab:mech_cassan}
  \begin{center}
    \begin{tabular}{c}
    \panel{A. Father-son pairs} \\
    \setlength{\linewidth}{.1cm} \begin{center}
\newcommand{\contents}{\begin{tabular}{l*{3}{c}}
\hline\hline
                    &         (1)   &         (2)   &         (3)   \\
\hline
Post * Late SC      &       8.432***&       6.764***&               \\
                    &     (1.794)   &     (1.555)   &               \\
1970-79 * Late SC   &               &               &       6.739** \\
                    &               &               &     (2.025)   \\
1980-89 * Late SC   &               &               &       9.649***\\
                    &               &               &     (2.580)   \\
N                   &        4502   &        3746   &        4502   \\
r2                  &        0.32   &        0.34   &        0.32   \\
\hline
\multicolumn{4}{p{\linewidth}}{$^{*}p<0.10, ^{**}p<0.05, ^{***}p<0.01$} \\
\multicolumn{4}{p{\linewidth}}{\footnotesize \tablenote}
\end{tabular} }
\setbox0=\hbox{\contents}
\setlength{\linewidth}{\wd0-2\tabcolsep-.25em} \contents \end{center}
 \\
    \panel{B. Father-daughter pairs} \\
    \setlength{\linewidth}{.1cm} \begin{center}
\newcommand{\contents}{\begin{tabular}{l*{3}{c}}
\hline\hline
                    &         (1)   &         (2)   &         (3)   \\
\hline
Post * Late SC      &      -3.183   &      -1.604   &               \\
                    &     (1.645)   &     (1.666)   &               \\
1970-79 * Late SC   &               &               &      -2.658   \\
                    &               &               &     (1.727)   \\
1980-89 * Late SC   &               &               &      -1.685   \\
                    &               &               &     (2.342)   \\
N                   &        3429   &        3040   &        3177   \\
r2                  &        0.34   &        0.33   &        0.33   \\
\hline
\multicolumn{4}{p{\linewidth}}{$^{*}p<0.10, ^{**}p<0.05, ^{***}p<0.01$} \\
\multicolumn{4}{p{\linewidth}}{\footnotesize \tablenote}
\end{tabular} }
\setbox0=\hbox{\contents}
\setlength{\linewidth}{\wd0-2\tabcolsep-.25em} \contents \end{center}
 \\
    \end{tabular}
  \end{center}
  
  \footnotesize{Table~\ref{tab:mech_cassan} shows estimates from Equation~\ref{eq:cassan}, which describes the impact of Scheduled Caste designation on upward mobility. The dependent variable is the child education rank. The sample consists of SC children of fathers in approximately the bottom 60\% of the education distribution. \textit{Late SC} is an indicator for jati groups that were added to Scheduled Caste lists in the caste redesignation of 1977. \textit{Post} is an indicator for cohorts born after 1970. The sample in Panel A is all men, while Panel B is all women. In the Column 1 estimation, the are 696 ``late SC'' men, of whom 438 were born after 1970. For women, these numbers are 516 and 391 respectively. All estimations control for region $\times$ cohort, jati $\times$ region, jati $\times$ cohort, and birth year, and are clustered at the jati and the region levels. Source: IHDS (2012).}
\end{table}

