We study intergenerational mobility in India.  We propose a new measure of upward mobility: the expected education rank of a child born to parents in the bottom half of the education distribution. This measure works well under data constraints common in developing countries and historical contexts. Intergenerational mobility in India has been constant and low since before liberalization. Among sons, we observe rising mobility for Scheduled Castes and declining mobility among Muslims. Daughters' intergenerational mobility is lower than sons', with less cross-group variation over time. A natural experiment suggests that affirmative action for Scheduled Castes has substantially improved their mobility.
