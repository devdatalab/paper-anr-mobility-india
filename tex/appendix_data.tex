
This section describes the data sources and data construction in
detail. 

\subsection{IHDS}

The India Human Development Survey (IHDS) is a nationally
representative survey of 41,554 households, with rounds in 2004--05
and 2011--12. Definitions of social groups are described in the body
of the paper. This section focuses on linking parents to children.

The primary module of IHDS records the education of the father of the household
head. A secondary module, the women's survey, records the education of
the father and mother of the female respondent, as well as the father
and mother of her husband if she is married. The women's survey is
given to one or two women aged 15--49 in each household. Because of
the upper age restriction on the women's survey, the oldest daughter
in our sample is born in 1962; we therefore do not have any links from
mothers or links to daughters for the 1950--59 birth cohort.

Finally, we created additional parent-child links using information
from the relationship field in the household roster. Specifically, we
linked the household head to their children and parents. We linked the
spouse of the household head to their children. We linked
grandchildren of the household head to the child of the household only
in cases where there was no possible ambiguity about the parents of
the grandchildren. In cases with no possible ambiguity, we linked
nieces/nephews of the household head to brothers of the household
head. We did not link individuals on the basis of in-law
relationships, because of the ambiguity in the definition of the
sibling-in-law (i.e. sibling of spouse vs. spouse of sibling).

In many cases, a parent's education is recorded in multiple ways,
allowing us to cross-check the validity of the responses.  For
example, the household head's father's education may be obtained from
(i) the household roster (if he is coresident); (ii) from the
household head's response to the father education question; and (iii)
from his wife's responses to the husband's father's education
question. The average correlation between parent education measured
across different sources is 0.9. Appendix Table~\ref{tab:validate_eds}
shows that the discrepancies between measures are not correlated with
household characteristics.

\subsection{Data from other countries}

We refer in the paper to mobility data from several other
countries. Data from Denmark, Sweden, and Norway were generously
shared with us by \citeasnoun{boserup2014} and
\citeasnoun{bratberg2017}. Income mobility estimates for the U.S. were
drawn from \citeasnoun{chetty2014b} and
\citeasnoun{chetty2018}. Educational mobility estimates from the
U.S. were calculated from a parent-child education transition matrix
describing children in the 2005--2015 ACS and parents in the 2000
Census, from the data package of \citeasnoun{chetty2018}.

